\chapter{Conclusion}\glsresetall
This report attempted to find the net power reduction resulting from using an oblique \gls*{ww} for a given wavy wall wavelength $k_x\pz$,  $k_z\pz$, and corresponding  \gls*{ssl} forcing amplitude $\hat{W}\pz\ssl$. Although this result was already given by \sct, it is believed that their results ignored the effects of changing dissipation due to changing mean streamwise velocity profiles at low Reynolds numbers ($\gls*{ret}\approx200$). Therefore, this project sought to rectify this issue. However, it is believed that this attempt was not entirely successful and produced erroneous results when compared to expectations from \gls*{dns} of both the \gls*{ssl} and \gls*{ww} flows. Nonetheless, there were three major results that could be drawn from this project.

Firstly, in this report, we attempted to change the \gls*{ww} dissipation rate from the \gls*{ww} wall units ($+\gls*{wv}$) into the wall units of the reference channel flow ($+\gls*{zer}$ ). However, it was found that in order to do so it was necessary to know the ratio $\frac{\tau_{w,\gls*{wv}}}{\tau_{w,\gls*{zer}}}$. In the \gls*{ssl} flow, this was not a problem as the wall was flat and has no pressure drag, which means the total drag was equivalent to the friction drag. Hence, $\frac{\tau_{w,\gls*{sl}}}{\tau_{w,\gls*{zer}}}=1-\frac{P_\text{sav,\gls*{sl}} }{100\%}$. However, in the \gls*{ww} flow, we know there will be pressure drag due to the presence of pressure gradients as a result of the wall shape, as a result, we do not know how to express it as a function of other known variables in our system of equations. We can only guess that it is approximately equal to $1-\frac{P_\text{net,\gls*{wv}} }{100\%}$ based on \gls*{dns} results. Therefore, further research must be done in order to find or approximate this ratio, or to check if our assumption that the pressure drag is a small portion of total drag when compared to the reference flow is correct.

Secondly, we found that by modelling the mean streamwise velocity profile as two separate portions (a linear and a logarithmic portion), the net power reduction of the \gls*{ww} flow, given the parameters $k_x\pz$,  $k_z\pz$, and  $\hat{W}\pz\ssl$, is uniquely determined by the point at which the linear and logarithmic portion meet $y_{\times}^{+}$, and that this could likely be extended to other flows of drag reduction given some tweaks in the equation. Moreover, $y_{\times}^{+}$ is also uniquely determined by the vertical height shift $\Delta h$ of the logarithmic portion of the mean velocity profile. Ideally, we would be able to predict purely with $k_x\pz,$  $k_z\pz$, and  $\alpha\pz,$ the amplitudes of the  \gls*{ww} waves themselves. However, this is not possible with the current analysis which uses $\hat{W}\pz\ssl$ and the wavelengths to determine the amplitude of periodic pressure fluctuations. Instead of knowing $\alpha\pz$ \textit{a priori}, currently it must be found by matching it to the correct pressure amplitude  $\hat{P}\pw\ww$. This also requires further research like that of \textcite{ghebali2017} and \textcite{denison2015}, although the former is the only detailed study to be published.

Finally, the second point is subject to the ratio of wall friction drag in point one being correctly defined. Without knowing the ratio, the system of equations is not closed, and we are unable to determine the net power reduction of the \gls*{ww} flow. Therefore, we believe that knowing this ratio (exactly or approximately) is of importance to solving our central problem.

In all, with a guess of the approximation of the wall friction drag ratio, we did not find a net power reduction for the \gls*{ww} flow but rather a net power addition of $17.15\%$ to drive the flow. It is believed that this value is erroneous and further research into our assumptions and any human errors is needed before we can conclude the actual net power savings of the wavy wall flow at these Reynolds numbers where the differences in mean profiles are significant.
